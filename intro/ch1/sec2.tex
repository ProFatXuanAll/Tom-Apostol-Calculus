\section{Historical Background}\label{sec:1.1.2}

\begin{note}
  The birth of integral calculus occurred more than 2000 years ago when the Greeks attempted to determine areas by a process that they called the \textbf{\iindex{method of exhaustion}}.
  The essential ideas of this method are very simple and can be described briefly as follows:
  Given a region whose area is to be determined, we inscribe a polygonal region that approximates the given region and whose area we can easily compute.
  Then, we choose another polygonal region, which gives a better approximation, and we continue the process, taking polygons with more and more sides to exhaust the given region.
  It was used successfully by \href{https://en.wikipedia.org/wiki/Archimedes}{\iindex{Archimedes}} (287--212 B.C.) to find exact formulas for the area of a circle and a few other special figures.

  The development of the method of exhaustion beyond the point to which Archimedes carried it had to wait nearly 18th centuries until the use of algebraic symbols and techniques became a standard part of mathematics.
  The elementary algebra that is familiar to most high-school students today was completely unknown in Archimedes' time, and it would have been next to impossible to extend his method to any general class of regions without some convenient way of expressing rather lengthy calculations in a compact and simplified form.

  A slow but revolutionary change in the development of mathematical notations began in the 16th Century A.D.
  The cumbersome system of Roman numerals was gradually displaced by the \iindex{Hindu-Arabie characters} used today, the symbols \(+\) and \(-\) were introduced, and the advantages of the decimal notation began to be recognized.
  During this same period, the brilliant successes of the Italian mathematicians \href{https://en.wikipedia.org/wiki/Nicolo_Tartaglia}{\iindex{Niccolò Fontana Tartaglia}}, \href{https://en.wikipedia.org/wiki/Gerolamo_Cardano}{\iindex{Gerolamo Cardano}}, and \href{https://en.wikipedia.org/wiki/Lodovico_Ferrari}{\iindex{Lodovico de Ferrari}} in finding algebraic solutions of cubic and quartic equations stimulated a great deal of activity in mathematics.
  They encouraged the growth and acceptance of a new and superior algebraic language.
  With the widespread introduction of well-chosen algebraic symbols, interest was revived in the ancient method of exhaustion, and a large number of fragmentary results were discovered in the 16th Century by such pioneers as \href{https://en.wikipedia.org/wiki/Bonaventura_Cavalieri}{\iindex{Bonaventura Francesco Cavalieri}}, \href{https://en.wikipedia.org/wiki/Evangelista_Torricelli}{\iindex{Evangelista Torricelli}}, \href{https://en.wikipedia.org/wiki/Gilles_de_Roberval}{\iindex{Gilles Personne de Roberval}}, \href{https://en.wikipedia.org/wiki/Pierre_de_Fermat}{\iindex{Pierre de Fermat}}, \href{https://en.wikipedia.org/wiki/Blaise_Pascal}{\iindex{Blaise Pascal}}, and \href{https://en.wikipedia.org/wiki/John_Wallis}{\iindex{John Wallis}}.

  Gradually, the method of exhaustion was transformed into the subject now called \iindex{integral calculus}, a new and powerful discipline with a large variety of applications, not only to geometrical problems concerned with areas and volumes but also to problems in other sciences.
  This branch of mathematics, which retained some of the original features of the method of exhaustion, received its biggest impetus in the 17th Century, largely due to the efforts of \href{https://en.wikipedia.org/wiki/Isaac_Newton}{\iindex{Isaac Newton}} (1642--1727) and \href{https://en.wikipedia.org/wiki/Gottfried_Wilhelm_Leibniz}{\iindex{Gottfried Leibniz}} (1646--1716), and its development continued well into the 19th Century before the subject was put on a firm mathematical basis by such men as \href{https://en.wikipedia.org/wiki/Augustin-Louis_Cauchy}{\iindex{Augustin-Louis Cauchy}} (1789--1857) and \href{https://en.wikipedia.org/wiki/Bernhard_Riemann}{\iindex{Bernhard Riemann}} (1826--1866).
  Further refinements and extensions of the theory are still being carried out in contemporary mathematics.
\end{note}
