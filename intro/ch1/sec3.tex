\section{The Method of Exhaustion for the Area of a Parabolic Segment}\label{sec:1.1.3}

\begin{note}
  Before we proceed to a systematic treatment of integral calculus, it will be instructive to apply the method of exhaustion directly to one of the special figures treated by Archimedes himself.
  We define a curve as follows:
  If we choose an arbitrary point on the positive \(x\)-axis and denote its distance from \(0\) by \(x\), then the vertical distance from this point to the curve is \(x^2\).
  The vertical distance from \(x\) to the curve is called the ``\iindex{ordinate}'' at \(x\).
  The curve itself is an example of what is known as a \textbf{\iindex{parabola}}.
  The region bounded by the curve, the \(x\)-axis, and an arbitrary verticle line \(x = b\) is called a \textbf{\iindex{parabolic segment}}.

  Given a point \(b\) on the positive \(x\)-axis, the parabolic segment may be enclosed in a rectangle of base \(b\) and altitude \(b^2\).
  Examination of the figure suggests that the area of the parabolic segment is less than half the area of the rectangle.
  Archimedes made the surprising discovery that the area of the parabolic segment is exactly one-third of that rectangle;
  that is to say, \(A = \dfrac{b^3}{3}\), where \(A\) denotes the area of the parabolic segment.

  The method to find the area of the parabolic segment is simply this:
  We slice the figure into a number of strips and obtain two approximations of the region, one from below and one from above, by using two sets of rectangles.
  (We use rectangles rather than arbitrary polygons to simplify the computations.)
  The area of the parabolic segment is larger than the total area of the inner rectangles (those below the curve) but smaller than that of the outer rectangles (those above the curve).

  If each strip is further subdivided to obtain a new approximation with a larger number of strips, the total area of the inner rectangles \textbf{increases}, whereas the total area of the outer rectangles \textbf{decreases}.
  Archimedes realized that an approximation to the area within any desired degree of accuracy could be obtained by simply taking enough strips.

  Let us carry out the actual computations that are required in this case.
  For the sake of simplicity, we subdivide the base into \(n\) equal parts, each of length \(\dfrac{b}{n}\).
  The points of subdivision correspond to the following values of \(x\):
  \[
    0, \dfrac{b}{n}, \dfrac{2b}{n}, \dfrac{3b}{n}, \dots, \dfrac{(n - 1) b}{n}, \dfrac{nb}{n} = b.
  \]
  A typical point of subdivision corresponds to \(x = \dfrac{kb}{n}\), where \(k\) takes the successive values \(k = 0, 1, 2, 3, \dots, n\).
  At each point \(\dfrac{kb}{n}\) we construct the outer rectangle of altitude \(\pa{\dfrac{kb}{n}}^2\).
  The area of this rectangle is the product of its base and altitude and is equal to
  \[
    \pa{\dfrac{b}{n}} \pa{\dfrac{kb}{n}}^2 = \dfrac{b^3}{n^3} k^2.
  \]
  Let us denote by \(S_n\) the sum of the areas of all the outer rectangles.
  Then since the \(k\)th rectangle has area \(\dfrac{b^3}{n^3} k^2\), we obtain the formula
  \begin{equation}\label{eq:1.1.1}
    S_n = \dfrac{b^3}{n^3} \pa{1^2 + 2^2 + 3^2 + \dots + n^2}.
  \end{equation}
  In the same way, we obtain a formula for the sum \(s_n\) of all the inner rectangles:
  \begin{equation}\label{eq:1.1.2}
    s_n = \dfrac{b^3}{n^3} \pa{0^2 + 1^2 + 2^2 + \dots + (n - 1)^2}.
  \end{equation}
  This brings us to a very important stage in the calculation.
  Notice that the factor multiplying \(\dfrac{b^3}{n^3}\) in \cref{eq:1.1.1} is the sum of the squares of the first \(n\) integers:
  \[
    1^2 + 2^2 + \dots + n^2.
  \]
  The corresponding factor in \cref{eq:1.1.2} is similar, except that the sum has only \(n - 1\) terms.
  For a large value of \(n\), the computation of this sum by directly adding its terms is tedious and inconvenient.
  Fortunately, there is an interesting identity that makes it possible to evaluate this sum in a simpler way, namely,
  \begin{equation}\label{eq:1.1.3}
    1^2 + 2^2 + \dots + n^2 = \dfrac{n^3}{3} + \dfrac{n^2}{2} + \dfrac{n}{6}.
  \end{equation}
  This identity is valid for every integer \(n \geq 1\) and can be proved as follows:
  Start with the formula \((k + 1)^3 = k^3 + 3 k^2 + 3k + 1\) and rewrite it in the form
  \[
    3 k^2 + 3k + 1 = (k + 1)^3 - k^3.
  \]
  Taking \(k = 1, 2, \dots, n - 1\), we get \(n - 1\) formulas
  \begin{align*}
    3 \cdot 1^2 + 3 \cdot 1 + 1             & = 2^3 - 1^3        \\
    3 \cdot 2^2 + 3 \cdot 2 + 1             & = 3^3 - 2^3        \\
    \vdots                                  &                    \\
    3 \cdot (n - 1)^2 + 3 \cdot (n - 1) + 1 & = n^3 - (n - 1)^3.
  \end{align*}
  When we add these formulas, all the terms on the right cancel except two and we obtain
  \[
    3 \pa{1^2 + 2^2 + \dots + (n - 1)^2} + 3 \pa{1 + 2 + \dots + (n - 1)} + (n - 1) = n^3 - 1^3.
  \]
  The second sum on the left is the sum of terms in an arithmetic progression and it simplifies to \(\dfrac{1}{2} n (n - 1)\).
  Therefore this last equation gives us
  \begin{equation}\label{eq:1.1.4}
    1^2 + 2^2 + \dots + (n - 1)^2 = \dfrac{n^3}{3} - \dfrac{n^2}{2} + \dfrac{n}{6}.
  \end{equation}
  Adding \(n^2\) to both members, we otain \cref{eq:1.1.3}.

  For our purposes, we do not need the exact expressions given in the right-hand members of \cref{eq:1.1.3,eq:1.1.4}.
  All we need are the two inequalities
  \begin{equation}\label{eq:1.1.5}
    1^2 + 2^2 + \dots + (n - 1)^2 < \dfrac{n^3}{3} < 1^2 + 2^2 + \dots + n^2
  \end{equation}
  which are valid for every integer \(n \geq 1\).
  These inequalities can de deduced easily as consequences of \cref{eq:1.1.3,eq:1.1.4}, or they can be proved directly by induction.
  (A proof by induction is given in \cref{sec:1.4.1}.)

  If we multiply both inequalities in \cref{eq:1.1.5} by \(\dfrac{b^3}{n^3}\) and make use of \cref{eq:1.1.1,eq:1.1.2} we obtain
  \begin{equation}\label{eq:1.1.6}
    s_n < \dfrac{b^3}{3} < S_n
  \end{equation}
  for every integer \(n \geq 1\).
  The inequalities in \cref{eq:1.1.6} tell us that \(\dfrac{b^3}{3}\) is a number which lies between \(s_n\) and \(S_n\) for every integer \(n \geq 1\).
  We will now prove that \(\dfrac{b^3}{3}\) is the only number which has this property.
  In other words, we assert that if \(A\) is any number which satisfies the inequalities
  \begin{equation}\label{eq:1.1.7}
    s_n < A < S_n
  \end{equation}
  for every positive integer \(n\), then \(A = \dfrac{b^3}{3}\).
  It is because of this fact that Archimedes concluded that the area of the parabolic segment is \(\dfrac{b^3}{3}\).

  To prove that \(A = \dfrac{b^3}{3}\), we use the inequalities in \cref{eq:1.1.5} once more.
  Adding \(n^2\) to both sides of the leftmost inequality in \cref{eq:1.1.5}, we obtain
  \[
    1^2 + 2^2 + \dots + n^2 < \dfrac{n^3}{3} + n^2.
  \]
  Multiplying this by \(\dfrac{b^3}{n^3}\) and using \cref{eq:1.1.1}, we find
  \begin{equation}\label{eq:1.1.8}
    S_n < \dfrac{b^3}{3} + \dfrac{b^3}{n}.
  \end{equation}
  Similarly, by subtracting \(n^2\) from both sides of the rightmost inequality in \cref{eq:1.1.5} and multiplying by \(\dfrac{b^3}{n^3}\), we are led to the inequaiity
  \begin{equation}\label{eq:1.1.9}
    \dfrac{b^3}{3} - \dfrac{b^3}{n} < s_n.
  \end{equation}
  Therefore, any number \(A\) satisfying \cref{eq:1.1.7} must also satisfy
  \begin{equation}\label{eq:1.1.10}
    \dfrac{b^3}{3} - \dfrac{b^3}{n} < A < \dfrac{b^3}{3} + \dfrac{b^3}{n}
  \end{equation}
  for every integer \(n \geq 1\).
  Now there are only three possibilities:
  \[
    A > \dfrac{b^3}{3}, \quad A < \dfrac{b^3}{3}, \quad A = \dfrac{b^3}{3}.
  \]
  If we show that each of the first two leads to a contradiction, then we must have \(A = \dfrac{b^3}{3}\), since, in the manner of Sherlock Holmes, this exhausts all the possibilities.

  Suppose the inequality \(A > \dfrac{b^3}{3}\) were true.
  From the second inequality in \cref{eq:1.1.10} we obtain
  \begin{equation}\label{eq:1.1.11}
    A - \dfrac{b^3}{3} < \dfrac{b^3}{n}
  \end{equation}
  for every integer \(n \geq 1\).
  Since \(A - \dfrac{b^3}{3}\) is positive, we may divide both sides of \cref{eq:1.1.11} by \(A - \dfrac{b^3}{3}\) and then multiply by \(n\) to obtain the equivalent statement
  \[
    n < \dfrac{b^3}{A - b^3 / 3}
  \]
  for every integer \(n \geq 1\).
  But this inequality is obviously false when \(n \geq \dfrac{b^3}{A - b^3 / 3}\).
  Hence the inequality \(A > \dfrac{b^3}{3}\) leads to a contradiction.
  By a similar argument, we can show that the inequality \(A < \dfrac{b^3}{3}\) also leads to a contradiction, and therefore we must have \(A = \dfrac{b^3}{3}\), as asserted.
\end{note}
