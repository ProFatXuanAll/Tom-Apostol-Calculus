\section{A Critical Analysis of Archimedes' Method}\label{sec:1.1.5}

\begin{note}
  Every branch of knowledge is a collection of ideas described by means of words and symbols, and one cannot understand these ideas unless one knows the exact meanings of the words and symbols that are used.
  Certain branches of knowledge, known as \textbf{\iindex{deductive systems}}, are different from others in that a number of ``undefined'' concepts are chosen in advance and all other concepts in the system are defined in terms of these.
  Certain statements about these undefined concepts are taken as \textbf{\iindex{axioms}} or \textbf{\iindex{postulates}} and other statements that can be deduced from the axioms are called \textbf{\iindex{theorems}}.
  The most familiar example of a deductive system is the Euclidean theory of elementary geometry that has been studied by well-educated men since the time of the ancient Greeks.

  The spirit of early Greek mathematics, with its emphasis on the theoretical and postulational approach to geometry as presented in Euclid's Elements, dominated the thinking of mathematicians until the time of the Renaissance.
  A new and vigorous phase in the development of mathematics began with the advent of algebra in the 16th Century, and the next 300 years witnessed a flood of important discoveries.
  Conspicuously absent from this period was the logically precise reasoning of the deductive method with its use of axioms, definitions, and theorems.
  Instead, the pioneers in the 16th, 17th, and 18th centuries resorted to a curious blend of deductive reasoning combined with intuition, pure guesswork, and mysticism, and it is not surprising to find that some of their work was later shown to be incorrect.
  However, a surprisingly large number of important discoveries emerged from this era, and a great deal of the work has survived the test of history-attribute to the unusual skill and ingenuity of these pioneers.

  As the flood of new discoveries began to recede, a new and more critical period emerged.
  Little by little, mathematicians felt forced to return to the classical ideals of the deductive method in an attempt to put the new mathematics on a firm foundation.
  This phase of the development, which began early in the 19th Century and has continued to the present day, has resulted in a degree of logical purity and abstraction that has surpassed all the traditions of Greek science.
  At the same time, it has brought about a clearer understanding of the foundations of not only calculus but of all of mathematics.

  There are many ways to develop calculus as a deductive system.
  One possible approach is to take the real numbers as the undefined objects.
  Some of the rules governing the operations on real numbers may then be taken as axioms.
  One such set of axioms is listed in \cref{ch:1.3}.
  New concepts, such as integral, limit, continuity, derivative, must then be defined in terms of real numbers.
  Properties of these concepts are then deduced as theorems that follow from the axioms.

  Looked at as part of the deductive system of calculus, Archimedes' result about the area of a parabolic segment cannot be accepted as a theorem until a satisfactory definition of area is given first.
  It is not clear whether Archimedes had ever formulated a precise definition of what he meant by area.
  He seems to have taken it for granted that every region has an area associated with it.
  On this assumption he then set out to calculate areas of particular regions.
  In his calculations he made use of certain facts about area that cannot be proved until we know what is meant by area.
  For instance, he assumed that if one region lies inside another, the area of the smaller region cannot exceed that of the larger region.
  Also, if a region is decomposed into two or more parts, the sum of the areas of the individual parts is equal to the area of the whole region.
  All these are properties we would like area to possess, and we shall insist that any definition of area should imply these properties.
  It is quite possible that Archimedes himself may have taken area to be an undefined concept and then used the properties we just mentioned as axioms about area.

  Today we consider the work of Archimedes as being important not SO much because it helps us to compute areas of particular figures, but rather because it suggests a reasonable way to define the concept of area for more or less arbitrary figures.
  As it turns out, the method of Archimedes suggests a way to define a much more general concept known as the \textbf{integral}.
  The integral, in turn, is used to compute not only area but also quantities such as arc length, volume, work and others.

  The symbol \(\int\) (an elongated S) is called an \textbf{\iindex{integral sign}}, and it was introduced by Leibniz in 1675.
  The process which produces the number \(\int_a^b f(x) \; dx\) is called \textbf{\iindex{integration}}.
  The numbers \(a\) and \(b\) which are attached to the integral sign are referred to as the \textbf{\iindex{limits of integration}}.
  The symbol \(\int_a^b f(x) \; dx\) must be regarded as a whole.

  Leibniz' symbol for the integral was readily accepted by many early mathematicians because they liked to think of integration as a kind of ``summation process'' which enabled them to add together infinitely many ``infinitesimally small quantities.''
  For example, the area of the parabolic segment was conceived of as a sum of infinitely many infinitesimally thin rectangles of height \(x^2\) and base \(dx\).
  The integral sign represented the process of adding the areas of all these thin rectangles.
  This kind of thinking is suggestive and often very helpful, but it is not easy to assign a precise meaning to the idea of an ``infinitesimally small quantity.''
  Today the integral is defined in terms of the notion of real number without using ideas like ``infinitesimals.''
  This definition is given in \cref{ch:2.1}.
\end{note}
