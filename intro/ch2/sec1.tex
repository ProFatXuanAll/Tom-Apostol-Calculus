\section{Introduction to Set Theory}\label{sec:1.2.1}

\begin{note}
  \href{https://en.wikipedia.org/wiki/George_Boole}{\iindex{George Boole}} (1815--1864) was an English mathematician and logician.
  His book, \emph{An Investigation of the Laws of Thought}, published in 1854, marked the creation of the first workable system of symbolic logic.
  \href{https://en.wikipedia.org/wiki/Georg_Cantor}{\iindex{Georg Ferdinand Ludwig Philipp Cantor}} (1845--1918) and his school created the modern theory of sets during the period 1874--1895.
\end{note}

\begin{note}
  In discussing any branch of mathematics, be it analysis, algebra, or geometry, using the notation and terminology of set theory is helpful.
  This subject, which Boole and Cantor developed in the latter part of the 19th Century, has profoundly influenced the development of mathematics in the 20th Century.
  It has unified many seemingly disconnected ideas and helped reduce mathematical concepts to their logical foundations in an elegant and systematic way.
  A thorough treatment of the theory of sets would require a lengthy discussion, which we regard as outside the scope of this book.
  Fortunately, the basic notions are few, and it is possible to develop a working knowledge of the methods and ideas of set theory through an informal discussion.
\end{note}

\begin{defn}\label{defn:1.2.1}
  In mathematics, the word ``\textbf{\iindex{set}}'' is used to represent a collection of objects viewed as a single entity.
  The individual objects in the collection are called \textbf{\iindex{elements}} or \textbf{\iindex{members}} of the set, and they are said to \textbf{\iindex{belong to}} or to be \textbf{\iindex{contained in}} the set.
  The set, in turn, is said to \textbf{\iindex{contain}} or be \textbf{\iindex{composed of}} its elements.
\end{defn}

\begin{note}
  In many applications, it is convenient to deal with sets in which nothing special is assumed about the nature of the individual objects in the collection.
  These are called \textbf{\iindex{abstract sets}}.
  Abstract set theory has been developed to deal with such collections of arbitrary objects, and from this generality, the theory derives its power.
\end{note}
