\section{Introduction to Set Theory}\label{sec:1.2.1}

\begin{note}
  \iindex{George Boole} (1815--1864) was an English mathematician and logician.
  His book, \emph{An Investigation of the Laws of Thought}, published in 1854, marked the creation of the first workable system of symbolic logic.
  \iindex{Georg Ferdinand Ludwig Philipp Cantor} (1845--1918) and his school created the modern theory of sets during the period 1874--1895.
\end{note}

\begin{note}
  In discussing any branch of mathematics, be it analysis, algebra, or geometry, it is helpful to use the notation and terminology of set theory.
  This subject, which was developed by Boole and Cantort in the latter part of the 19th Century, has had a profound influence on the development of mathematics in the 20th Century.
  It has unified many seemingly disconnected ideas and has helped to reduce many mathematical concepts to their logical foundations in an elegant and systematic way.
  A thorough treatment of the theory of sets would require a lengthy discussion which we regard as outside the scope of this book.
  Fortunately, the basic notions are few in number, and it is possible to develop a working knowledge of the methods and ideas of set theory through an informal discussion.
  Actually, we shall discuss not so much a new theory as an agreement about the precise terminology that we wish to apply to more or less familiar ideas.
\end{note}

\begin{defn}\label{defn:1.2.1}
  In mathematics, the word ``\textbf{\iindex{set}}'' is used to represent a collection of objects viewed as a single entity.
  The collections called to mind by such nouns as ``flock,'' ``tribe,'' ``crowd,'' ``team,'' and ``electorate'' are all examples of sets.
  The individual abjects in the collection are called \textbf{elements} or \textbf{members} of the set, and they are said to \textbf{belong to} or to be \textbf{contained in} the set.
  The set, in turn, is said to \textbf{contain} or be \textbf{composed of} its elements.
\end{defn}

\begin{note}
  In many applications it is convenient to deal with sets in which nothing special is assumed about the nature of the individual abjects in the collection.
  These are called \textbf{\iindex{abstract sets}}.
  Abstract set theory has been developed to deal with such collections of arbitrary abjects, and from this generality the theory derives its power.
\end{note}
