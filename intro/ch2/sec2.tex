\section{Notations for Designating Sets}\label{sec:1.2.2}

\begin{defn}\label{defn:1.2.2}
  Sets usually are denoted by capital letters:
  \(A, B, C, \dots, X, Y, Z\);
  elements are designated by lower-case letters:
  \(a, b, c, \dots, x, y, z\).
  We use the special notation
  \[
    x \in S
  \]
  to mean that ``\(x\) is an element of \(S\)'' or ``\(x\) belongs to \(S\).''
  If \(x\) does not belong to \(S\), we write \(x \notin S\).
  When convenient, we shall designate sets by displaying the elements in braces;
  The dots are used only when the meaning of ``and so on'' is clear.
  The method of listing the members of a set within braces is sometimes referred to as \textbf{\iindex{the roster notation}}.
\end{defn}

\begin{defn}[\iindex{set equality}]\label{defn:1.2.3}
  Two sets \(A\) and \(B\) are said to be \textbf{equal} (or identical) iff they consist of exactly the same elements, in which case we write \(A = B\).
  If one of the sets contains an element not in the other, we say the sets are \textbf{unequal} and we write \(A \neq B\).
\end{defn}
